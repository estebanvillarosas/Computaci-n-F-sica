\documentclass[letterpaper,12pt]{article}

%----------Idioma---------
\usepackage[Spanish]{babel}

%---------Margenes--------
\usepackage[right=3cm,left=3cm,top=2cm,bottom=2cm]{geometry}

%--------Para poner símbolos especiales----------
\usepackage[utf8]{inputenc}

%-------Colores--------
\usepackage[usenames,dvipsnames,svgnames,table]{xcolor}

\title{\Huge{Una parte de mi vida}}
\author{\normal{Esteban Villa Rosas }}
\date{\normlal{September 12, 2022}}

\begin{document}

\maketitle

\section{\huge{Academia}}
    \subsection{\Large{Pasado}} \large{Vengo de la Escuela Nacional Preparatoria Número 4, ``Vidal Castañeda y Nájera''. La verdad es que esa etapa de bachillerato fue muy bonita, pues a pesar de que estuvimos dos años en línea pude convivir con algunas personas, por ejemplo cuando haciamos tarea en equipo, era divertido ver a los integrantes, que en ocaciones jugaban con los fondos de Zoom o hasta se les cortaba la voz (en ocaciones también me paso). Pero creo que lo que permitió esa convivencia, fue principalmente que los integrantes de mis equipos eran amigos que conocí en presencial, y con quien me llevava muy bien, pues en presencial siempre estabamos juntos, ibamos a canchas, a comer y por lo regular caminabamos juntos hacia el transporte.
    
    Por otra parte, en cuanto a distancia y a tiempo que realizaba hacia la prepa, sinceramente era bastante, bueno a mi parecer, pues si vivía muy lejos. Por lo regular hacía dos horas y media desde mi casa hasta la prepa, ya que vivía en un pueblito cerca de Toluca y tenía que ir y venir diario. Y bueno, el recorrido que hacía si estaba complicado, de hecho muchos de mis amigos decían que peregrinaba, y era cierto, pues debía caminar 10 minutos de mi casa al taxi, me bajaba del taxi y caminaba otros 10 minutos para tomar un camión que me dejaba en periférico, de ahí me subía a una combi que me llevava a Tacubaya y una vez estando en  Tacubaya, caminaba 12 minutos para llegar a la prepa.}
    
    \subsection{\Large{Actualidad}} \large{En principio, cabe mencionar que actualmente junto con mi hermano estamos rentando un departamento en Lindavista y al menos yo pienso que está bien el tiempo que me hago hasta la H. Facultad de Ciencias, pues no está tan cerca pero tampoco tan lejos como lo era en la prepa. 
    
    Me estoy haciendo 1 hora, camino 6 minutos del departamento hasta la línea 5 del metro, me subo en Politécnico y bajo en la Raza, posteriormente me dirijo hacia la línea 3 (La verdad es que este tramo de la Raza a la línea tres, no me gusta pues está demaciado largo, debo caminar aproximadamente 10 minutos), y en esta línea ya bajo hasta universidad, de ahí ya solo camino 5 minutos para llegar a la H. Facultan de Ciencias.}
    
    \subsection{\Large{¿Por qué Física?}} \large{Bueno, pues es cierto que es la pregunta favorita, ya que siempre me lo pregunto a mi mismo, y está genial poderme responer, es interesante que cuando me hago esta pregunta, me gusta responderla en un lugar tranquilo, como lo es el bosque ya que puedo hablar tranquilamente conmigo mismo. Pero ahora les comparto mi respuesta. 
    
    A lo largo de mi niñez siempre me llamaba mucho la atención escuchar que los físicos y físicas realizaban cosas muy interesantes, o que nos podían proporcionar información sobre nuevos planetas o estrellas, así como también agujeros negros en el universo, yo al escuchar esto de niño, me emocionaba bastante, y decía, algún día me gustaría ser quien descubra algo ahí en ese universo, por que sinceramente, a mi si me gustaría dedicarme a la investigación en el ámbito astrofísico.
    
    Pero bueno, eso no es todo, también lo que me motivó a la Física, fue siempre estar pensando el cómo los carros aceleraban, se paraban, y volvían a acelerar, hasta este punto ya es algo que logro comprender, pero cuando estaba chiquito siempre fue como muy interesante para mí, y me divertía viendo cómo las personas cuando frenaba el auto se iban hacia enfrente y cuando aceleraban hacia atras, aunque cuando a mi me pasaba esto, me enojaba. Y así fue la manera en que me fué interesando cada vez más la Física y mirenme, estoy aquí tratando y dando todo de mi, para lograr este gran sueño.}
    
\section{\huge{Hobbies o pasatiempos }}
    \subsection{\Large{Charrería}} \large{La verdad es que la charrería me encanta, y es algo que practico desde que era un niño, pues yo me crié en un ranchito y sinceramente me encanta el poder convivir con los animales, en este caso más con los caballos, ellos siempre están para mí y yo para ellos, es por eso que que amo este deporte de la charrería, que además ya es patrimonio cultural inmaterial de la humanidad.
    
    Por otra parte, cuando practico charrería, me relaja demasiado, me olvido de todo y me desestreso. No se si en alguna ocasión ustedes hayan montado a caballo, pero pues esto es muy agradable y más cuando lo haces con gusto.
    
    Así mismo, esta gran actividaad, me ayuda mucho a mantenerme bien físicamente, por que aunque muchas personas no lo crean, uno al estar compitiendo o entrenando se desgasta demasiado físicamente, de hecho yo hago ejercicio para la charrería, es decir para tener un mejor rendimiento dentro de esta disciplina, esto es por otra de las razones que me gusta. En lo particular, yo realizo lo que es el floreo de soga tanto a pie como a caballo, y mis suertes dentro del deporte, son las manganas a pie y la terna, que es referente a lo que mejor se hacer que es el floreo.
    
    En vacaciones, por lo regular le dedicaba catorce horas semanales a este deporte, es decir, dos horas por día, y claro era después de terminar mis obligaciones y de trabajar, pero ahora que ya estamos en clases, solo le dedico 5 horas semanales, y solo es el sábado, que es cuando más tiempo libre tengo. Cabe mencionar que actualmente soy integrante del equipo representativo de charreria dentro de la UNAM, y encerio que esto me enorgullese demaciado, pues es bonito el poder representar a tu universidad en algo que de verdad te apasiona, he aquí el por qué le dedico tanto tiempo a entrenar y a las competencias, pues siempre he querido mejorar más para poder dejarle algún reconocimiento o trofeo a mi universidad.} 

          
    \subsection{\Large{Leer libros sobre el universo}} \large{Realmente soy fan de todo lo que pueda englobar el universo, es por eso que me gusta demaciado leer este tipo de libros. Tal vez parezca un poco fuera de lo normal, pero al estar leyendo me siento como si fuera yo quien viajara al rededor de los planetas, las estrellas y hasta a un lado de la luz y el tiempo, me apaciona poder imaginar que estoy en otra dimensión saliendome de mi realidad misma, pues me concentro tanto en saber que va a pasar o que descubrire en los diferentes libros que me olvido de lo que pasa a mi alrededor, y esto es por lo que me encanta este pasatiempo, pues al igual que el anterior, puedo dejar de pensar en la presión de las cosas que tengo por hacer y me relajo introduciendome en el universo.
    
    Por lo regular le dedico 5 horas semanales a leer, pues a pesar de que me gusta demaciado, no tengo el tiempo suficiente para leer más, así que aprovecho el tiempo que voy en el metro para poder leer, o en los cambios de clase. Pero realmente si tuviera más tiempo, lo pasaría leyendo estos hermosos libros. 
    
    En este momento estoy leyendo ``El Universo en tu Mano'' y me apaciona demaciado, pues el autor nos introcuce en un viaje extraordinario a los límites del tiempo y el espacio. Así que espero tener un poco más de tiempo para terminarlo pronto.}
    
\section{\huge{Géneros de música preferida}}
    \subsection{\Large{Ranchera}} \large{Me gusta demasiado escuchar este tipo de música, ya sea cuando estoy recogiendo mi cuarto, o igual cuando estoy descansando, y de esta manera canto y me desahogo un poco, ya que al menos yo pienso que la música ranchera te ayuda a desahogarte de muchas cosas, pues ocupa mucho "corazón" para ser cantada. También, me gusta escucharla cuando salgo con mi hermano o mis amigos a montar a caballo, o hasta en una charreada, aunque también en familia, cuando estamos conviviendo, pues tenemos eso en común, y realmente es muy bonito. Algunos de los ejemplos de canciones de este género musical que me gustan son los siguientes:
    
    - ``Hoy platiqué con mi gallo'', canción de Vicente Fernández.
    
    - ``Ya la luna va saliendo'', canción de Antonio Aguilar.
    
    - ``México lindo y querido'', canción de Jorge Negrete.
    
    - ``Cien años'', canción de Pedro Infante.}

    \subsection{\Large{Rock clásico}} \large{Bueno, pues este género de música también me gusta demaciado, y por lo regular suelo escucharlo más cuando voy caminando para la escuela, aunque también, puede ser cuando hago la tarea o me estoy bañando. Sinceramente esta música, en lo personal me relaja demaciado, y me tranquiliza para pensar mejor algunas cosas. Algunos ejemplos de las canciones que me apasionan de este género son:
    
    - ``Fué en un café'', canción de Kenny Young y Artie Resnick.
    
    - ``Devuélveme a mi chica'', canción de Hombres G.
    
    - ``Have you ever seen the rain?'', canción de Creedence Clearwater.}
    
\section{\huge{Anécdotas o datos curiosos}} 
    \subsection{\Large{Acción}} \large{En mi ranchito siempre eh tenido borregos, entonces en una ocasión cuando tenía 8 años, a mi me gustaba demasiado corretear a los borregos, y pues por miedo de que los correteaba se alejaban, pero después de cierto tiempo, cómo que los borregos, se pusieron de acuerdo y ahora fue al revés, yo recuerdo que iba a corretearlos, pero cuando justo iba a hacerlo, todo el rebaño se fué tras de mí, y la verdad en ese instante no sé si me dió miedo o me dió risa, pero en mi se despertó una adrenalina súper fuerte, que corrí lo más rápido que pude, y lo más chistoso fue que no dejaban de seguirme, hasta que logre trepar un árbol y ahí me espere a que se fueran.}

    \subsection{\Large{Gracioso}} \large{Recuerdo mucho que mi hermano y yo jugábamos a menudo en la arena o en la tierra, entonces siempre terminamos bien sucios, y cómo todos sabemos nuestras mamás nos regañan por eso cuando somos pequeños, entonces en una ocasión, realmente si terminamos muy socios, y entramos a casa y ensuciamos el piso, entonces, mi mamá nos correteaba con una chancla, al rededor de la mesa, cabe mencionar que la mesa era grande, entonces nosotros solo nos reíamos, y corríamos, hasta que mi mamá, logró darle a mi hermano en la espalda con la chancla, y se cayó, entonces yo me puse en una esquina, para fingir que no hice nada. Esto cada que lo recuerdo me da mucha risa, y quería contarlo.}
    
    \subsection{\Large{Interesante}} \large{Algo muy sorprendente de mi, es que tal vez me vean y piensen que soy muy sociable, o que hablo demaciado, pero no es así, por lo regular, no soy de platicar con muchas personas, y cuando lo hago me pongo muy nervioso o simplemente me da pena, y bueno, esto me pareció curioso mencionarlo, porque al entrar a la Facultad de Ciencias, muchas personas me lo dijeron, que yo me veía muy social, pero realmente no es así, de hecho hay ocasiones que me da miedo hablarle a alguien.}
    
\section{\huge{Sección de puntos extra}}
    \subsection{\Large{Información de agrado}} \large{\textit{{\textbf{Aunque pueda ser sorprendente, el músculo que más utilizamos de todo nuestro cuerpo diariamente son los ojos. De hecho, el ojo puede moverse hasta 100.000 veces cada día, lo que podría ser el equivalente a caminar 80 kilómetros. La comparación anterior es necesaria para entender la magnitud del esfuerzo que se hace todos los días, especialmente porque no nos damos cuenta de todas las veces que parpadeamos. Los movimientos musculares de los ojos es difícil equipararlo con los músculos de otras partes de nuestro cuerpo.}}}

    \textcolor{red}{Lo más interesante es que pese a que no nos damos cuenta, los ojos también sufren de cansancio y de fatiga, especialmente cuando hacemos un esfuerzo por observar un punto determinado.} \textcolor{blue}{Además, el ojo es uno de los órganos más importantes del cuerpo porque es el encargado de recibir los rayos luminosos procedentes de los objetos y pasar toda información al cerebro.}}

\subsection{\Large{Mejor manga de la historia, ¿Griffith hizo algo malo?}}\large{El mejor manga de la historia es...} \huge{Slam Dunck}. \large{Por otra parte yo creo que Griffith si hizo algo malo, ya que a pesar de que se sienta traicionado tanto por Guts como por Casca, no debe de causarles daño emocional, porque es su reputación, si ya lo traicionaron, el no debe ser igual que ellos.} 
\end{document}
